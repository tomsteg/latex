\documentclass[11pt,a4paper]{scrartcl}
% a4paper bestimmt die Papiergröße als Din A4
% 11pt legt die Schriftgröße auf 11pt fest
% scartcl ist die KOMA Sckript Klasse für article; der größte Unterschied besteht darin, dass die Überschriften in einer serifenlosen Schrift dargestellt werden.
\usepackage{ngerman} % bestimmt die deutsche Sprache mit den neuen Rechtschribung als Grundlage für die Wörterbücher und die Metawörter
\usepackage{eurosym} % ermögicht mit \euro den Zugriff auf ein dem Zeichensatz angepasstes Eurosymbol
\usepackage{textcomp} % Zugriff auf zusätzliche Textsymbole, zum Beispiel \texteuro, \textcopyright, \textregistered, \texttrademark, \textdollar, \texteuro, \textperthousand
\usepackage{fontenc} % lädt die Textfontkodierung; dies wird benötigt um die utf Kodierung verwenden zu können
\usepackage[utf8]{inputenc} % ermöglicht die Verwendung von utf8-kodierten Quelltexten
\usepackage{alltt} % definiert die alltt-Umgebung, eine verbatim-Umgebung, in der die Zeichen \,{,} ihre Gültigkeit behalten
\usepackage{graphicx} % ermöglicht mit dem \includegraphics Befehl das Einbinden von externen Graphiken
\usepackage{paralist} % erweitert die enumerate-Umgebung um einen optionalen Parameter, der die Art der Nummerierung angiebt: a), A., i), 1), Beispiel i), ...
\usepackage{tabularx} % ermöglicht die Verwendung von variablen Breiten mit der Formatanweisung X statt p{Breite} in der tabular-Umgebung
\usepackage{float} % der Parameter H in den Positionsoptionen von Gleitobjekten erzwingt die Positionierung an der Stelle im Quelltext
\usepackage{multicol} % ermöglicht die Verwendung des Befehls twocolumn auch in minipage oder figure Umgebungen, beziehungsweise die Verwendung der Umgebung multicols mit den Parametern {Spalten}[Vortext][Abstand]
\usepackage{fancyhdr} % ermöglicht die Verwendung eigener Kolumnentitel (Kopf- und Fusszeilen)
\usepackage{hyperref} % verwandelt alle Querverweise in Hyperlinks; es sollte als letztes Paket eingebunden werden

\begin{document}

\graphicspath{{./images/}} % gibt den Pfad zu den Graphikdateien an, die eingebunden werden; hier das Unterverzeichnis bilder
\DeclareGraphicsExtensions{.jpg,.png} % erlaubte Graphikobjekte

\title{Titel\\ \small Untertitel}
\author{Thomas Steglich\\
\small \url{www.thomas-steglich.de}\\
\small @tomsteg\\
\small info@thomas-steglich.de\\
\small 0151 24104316}
\date{\today}

\maketitle

\begin{abstract}
\noindent Text der Zusammenfassung
\end{abstract}

\pagestyle{fancy} % Layout Stil für Kopf- und Fusszeilen
\lhead{\leftmark} % linker,
\chead{} % mittlerer und
\rhead{\rightmark} % rechter Bereich der Kopfzeile
\lfoot{Thomas Steglich} % linker,
\cfoot{\thepage} % mittlerer und
\rfoot{\today} % rechter Bereich der Fusszeile, wobei thepage die Seitennummer ausgibt, und leftmark und rightmark die Kapitel und Unterkapitel ausgibt.
\renewcommand\headrulewidth{1pt} % gibt die Dicke der Linie zwischen Kopfzeile und Text an
\renewcommand\footrulewidth{1pt} % gibt die Dicke der Linie zwischen Text und Fusszeile an

\newpage % neue Seite für Inhaltsverzeichnis
\tableofcontents % Inhaltsverzeichnis
\newpage % neue Seite nach Inhaltsverzeichnis

\section{Überschrift}

Das ist ein Beispieltext\cite[Text]{zitierschluessel}.

\nocite{*} % alle Literaturangaben werden genannt, auch wenn keine Verweise auf sie zeigen
\bibliographystyle{plain}
\begin{thebibliography}{\hspace{1cm}}
	\bibitem [label]{zitierschluessel} bibliographische Information
\end{thebibliography}

\end{document}
